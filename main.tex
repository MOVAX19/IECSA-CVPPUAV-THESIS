\documentclass[oneside,numbers,spanish]{ezthesis}
\usepackage[spanish]{babel}
\usepackage{amssymb}
\usepackage{here}
\usepackage{graphics}
\usepackage{amsmath}
\usepackage{fancyhdr}
\usepackage{longtable}
\usepackage{multirow} %unir renglones
\usepackage[utf8]{inputenc}
\usepackage{subfig}
\usepackage{multirow}
\usepackage{amsmath}
\usepackage{mathrsfs}
\usepackage{amsfonts}
\usepackage{dsfont}
\usepackage{amssymb}
\usepackage{graphicx}
\usepackage{pdfpages}
\usepackage{multirow}
\usepackage[Conny]{fncychap}
\usepackage{xcolor}
\usepackage{xpatch}
\usepackage{listings}
\usepackage{subfig} % for subfigures
\usepackage{caption}
%\usepackage{merriweather} %% Option 'black' gives heavier bold face 
%\usepackage[rm]{roboto}
%\usepackage[T1]{fontenc}
%\usepackage[sfdefault,scaled=.85]{FiraSans}
\usepackage{libertine}
%\usepackage{libertinust1math}
\usepackage{newtxsf}
\usepackage{url}
\usepackage[hidelinks]{hyperref}
\urlstyle{same}
\spanishdecimal{.}
\renewcommand\spanishtablename{Tabla}
\renewcommand\spanishlisttablename{\'{I}ndice de tablas}
\renewcommand{\bibname}{\textbf{Referencias }}
\renewcommand{\tablename}{Tabla}
\renewcommand{\listtablename}{\'{I}ndice de tablas}
\renewcommand{\max}{max}
\renewcommand{\min}{min}
\graphicspath{ {Images/} }
%\definecolor{UNAQ}{HTML}{00335F}
\definecolor{UNAQ}{HTML}{00437e}
\definecolor{mygreen}{rgb}{0,0.6,0}
\definecolor{mygray}{rgb}{0.5,0.5,0.5}
\definecolor{mymauve}{rgb}{0.58,0,0.82}



\newtheorem{definition}{Definici\'{o}n}[section]
\newtheorem{theorem}{Teorema}[section]
\newtheorem{lemma}[theorem]{Lema}
\newtheorem{proposition}[theorem]{Proposici\'{o}n}
\newtheorem{corollary}[theorem]{Corolario}
\newtheorem{comment}[theorem]{Comentario}
\newtheorem{remark}[theorem]{Nota}

%Page header formatting
\pagestyle{fancy}
%Normal pages style
\renewcommand{\headrule}{\hbox to\headwidth{%
  \color{UNAQ}\leaders\hrule height \headrulewidth\hfill}}
\setlength{\headheight}{26pt}
\renewcommand{\headrulewidth}{1.5pt}
\fancyhead[R]{}
\fancyfoot[R]{\thepage}

%Style for chapter headers
\fancypagestyle{plain}{%
  \fancyhf{}
  \fancyhead[L]{\includegraphics[width=3.5cm]{UNAQBlue.pdf}}
  \fancyfoot[R]{\textbf{\thepage}} 
  \renewcommand{\headrulewidth}{0pt}
  \renewcommand{\footrulewidth}{0pt}
  \setlength{\headheight}{56pt}
}


%Chapter title formatting
\ChNameUpperCase 
\ChTitleUpperCase 
\ChNameVar{\centering\Huge\rm\bfseries\color{black}}
\ChNumVar{\Huge\color{black}} 
\ChRuleWidth{2pt} 
\ChTitleVar{\centering\Huge\rm\color{black}}

\xpatchcmd\DOCH
  {\mghrulefill}{\color{UNAQ}\mghrulefill}
  {}{\PatchFailed}
\xpatchcmd\DOTI
  {\mghrulefill}{\color{UNAQ}\mghrulefill}
  {}{\PatchFailed}
\xpatchcmd\DOTIS
  {\mghrulefill}{\color{UNAQ}\mghrulefill}
  {}{\PatchFailed}

\usepackage{listings}
\lstset{ %
language=bash,                % choose the language of the code
basicstyle=\footnotesize,       % the size of the fonts that are used for the code
%numbers=left,                   % where to put the line-numbers
%numberstyle=\footnotesize,      % the size of the fonts that are used for the line-numbers
stepnumber=1,                   % the step between two line-numbers. If it is 1 each line will be numbered
numbersep=5pt,                  % how far the line-numbers are from the code
backgroundcolor=\color{white},  % choose the background color. You must add \usepackage{color}
showspaces=false,               % show spaces adding particular underscores
showstringspaces=false,         % underline spaces within strings
showtabs=false,                 % show tabs within strings adding particular underscores
%frame=single,           % adds a frame around the code
tabsize=2,          % sets default tabsize to 2 spaces
captionpos=b,           % sets the caption-position to bottom
breaklines=true,        % sets automatic line breaking
breakatwhitespace=false,    % sets if automatic breaks should only happen at whitespace
escapeinside={\%*}{*)},          % if you want to add a comment within your code
commentstyle=\color{mygreen},    % comment style
keywordstyle=\color{blue},       % keyword style
numberstyle=\tiny\color{mygray},
stringstyle=\color{mymauve}
}


%\hyperlinking
\begin{document}



\includepdf[pages=1-1]{PORTADA.pdf}
%\chapter{Conny}
\chapter*{Resumen}
En el presente trabajo se integra un conjunto de software de código libre con el objetivo de proponer e implementar un ambiente de simulación para software in the loop, que permita la validación de un sistema de misión de vuelo de un quadrotor autónomo virtual; el sistema en cuestión se encuentra integrado por dos grandes partes, por un lado, se tiene un algoritmo de visión por artificial para la detección de un tipo de compuerta utilizado en las competencias de drones autónomos para delimitar circuitos de vuelo, y por otro lado, se implementa un algoritmo de seguimiento de trayectoria basado en waypoints, en donde un script se comunica con un firmware de piloto automático, de tal forma que se envían comandos de vuelo específicos para que el dron sea capaz pasar a través de una serie de compuertas y completar la trayectoria que define el circuito vuelo.

Así mismo, dentro de la simulación se pueden destacar dos componentes de gran relevancia, el circuito de vuelo y el dron virtual. La figura \ref{fig:method} presenta un esquema de alto nivel del sistema que se implementa en la simulación. 

El circuito de vuelo se describe con 4 compuertas posicionadas en un arreglo rectangular simple y el dron virtual cuenta con una cámara simulada, a partir de la cual se extrae la trama de video para el algoritmo de segmentación, y un plugin para el piloto automático que le permite recibir comando de vuelo para volar a través del circuito.

\begin{figure}[ht]
    \centering
    \includegraphics[width=0.8\textwidth]{Method.pdf}
    \caption{Diagrama general de los componentes de la simulación}
    \label{fig:method}
\end{figure}

Adicionalmente, y como se mencionó anteriormente, el sistema de misión de vuelo se conforma por un conjunto de software y librerías de última generación, las cuales se describen a continuación.

\textbf{OpenCV}: una librería robusta para aplicaciones de visión artificial, con la cual se llevó a cabo la implementación del algoritmo de detección de compuertas.

\textbf{PymavLink}: una librería que cuenta con una API basada en el protocolo de comunicación MAVLink. Esta permitió la comunicación con el firmware de piloto automático, ArduPilot.

\textbf{Gazebo}: un ambiente de simulación con gráficas en 3D, pensado para aplicaciones del área de robótica, en él se elaboró el circuito de vuelo. 

\textbf{ArduPilot}: un firmware para pilotos automáticos que ofrece herramientas para realizar simulación de software in the loop.

\textbf{ROS 2}: la nueva versión del framework de desarrollo de aplicaciones de robótica; parte esencial para la integración del sistema de misión de vuelo.

\textbf{GNU/Linux}:  el sistema operativo en donde se desarrolló el proyecto en su conjunto.


Por último, se documenta con gran detalle el proceso de integración de todas las herramientas de software utilizadas, así como el comportamiento final del sistema propuesto.











\chapter*{Glosario}

\textbf{  \normalsize Terminología}
\begin{description}
  \item[\textbf{Algoritmo:}] conjunto finito y ordenado de instrucciones que representan la solución a un problema.
  \item[\textbf{Aprendizaje profundo:}] del inglés '\textit{Deep Learning}', es una subárea de la inteligencia artificial, en donde el modelo de aprendizaje se basa en una gran conjunto de capas (de entrada, salidas y ocultas) compuestas por redes neuronales artificiales. Cada capa se especializa una tarea de predicción específica, de tal forma que una máquina es capaz de aprender por sí misma, sin necesidad de intervención humana.
  \item[\textbf{Arquitectura:}] dentro del campo de estudio de la inteligencia artificial, se refiere a las conexiones o el patrón de diseño de una red neuronal artificial.
  \item[\textbf{Código abierto:}] también conocido como \textit{software libre}, es un modelo de desarrollo de software que se fundamenta en la colaboración abierta, en donde cualquier usuario tiene la liberta de ejecutar, copiar, distribuir, modificar y contribuir a la mejora del software. 
  \item[\textbf{Comando de vuelo:}] dentro del contexto del firmware para pilotos automáticos, se refiere a instrucciones de alto nivel para que el piloto automático lleve a la aeronave a un estado deseado, dígase actitud, rumbo, etc.
  \item[\textbf{Convolución:}] operador matemático que representa la integral del producto de dos funciones, en donde una de las señales se encuentra trasladada e invertida. 
  \item[\textbf{Entrenamiento:}] en el campo de estudio de la inteligencia artificial, se refiere al conjunto de métodos a partir de los cuales una máquina es capaz de aprender.
  \item[\textbf{Espacio de color:}] también conocido como \textit{modelo de color}, se refiere al modelo matemático utilizado para describir los distintos sistemas mediante los cuales se pueden representar los colores, a partir de arreglos de 3 o 4 parámetros, generalmente.
  \item[\textbf{Firmware:}] es el software base que viene incluido en los dispositivos electrónicos o hardware, y se encarga de asegurar un funcionamiento básico correcto. También es conocido como \textit{soporte lógico inalterable}.
  \item[\textbf{Fotograma:}] cada una de las imágenes fijas, que en su conjunto forman una imagen en movimiento o video.
  \item[\textbf{Framework:}] en español \textit{entorno de trabajo}, es una estructura que integra tecnologías, estándares y módulos de software que sirve como base para el desarrollo software.
  \item[\textbf{Hardware:}] corresponde a los recursos físicos que componen o integran un equipo de cómputo o dispositivo lógico.
  \item[\textbf{Hardware in the loop:}] paradigma de simulación para la validación de sistemas embebidos en donde la planta que se desea controlar se simula a partir de un modelo matemático, mientras que el sistema de control es físico e interactúa de forma directa con la simulación de la planta.
  \item[\textbf{Histograma de color:}] es la cuantificación de la distribución de color en una imagen, generalmente se representa con una gráfica en donde se observa la frecuencia de pixeles del mismo color.
  \item[\textbf{Interfaz de programación de aplicación:}] del inglés \textit{Application} \textit{Programming} \textit{Interface}; se trata de un conjunto de definiciones y protocolos que permiten la comunicación o integración entre diferentes aplicaciones de software.
  \item[\textbf{Librería:}] también conocidas como \textit{bibliotecas}, es un conjunto de módulos o métodos funcionales de software, que fueron codificados para ofrecer una funcionalidad especifica y bien definida.
  \item[\textbf{Machine learning:}] conocido en español como \textit{aprendizaje automático}, es una subárea del campo de la inteligencia artificial en donde se implementa modelo matemático para que un sistema sea capaz de aprender a partir del procesamiento de datos sin la necesidad de especificar una programación explicita.
  \item[\textbf{Máquina de estados:}] es un modelo que describe el comportamiento de un sistema a partir de una serie de estados finitos, en donde la transición entre cada uno depende de la entrada actual del proceso y la o las entradas anteriores.
  \item[\textbf{Matiz:}] en el modelo de color HSV, corresponde a un ángulo dentro del rango de 0 a 360 grados, en donde cada grado está asociado a una tonalidad de color en específico.
  \item[\textbf{Multiplataforma:}] dicho de una aplicación de software que se encuentra disponible para su ejecución en distintos sistemas operativos o sistemas.
  \item[\textbf{Odometría:}] es el área que se encarga del estudio de la estimación de posición de cualquier tipo de vehículo durante su navegación.
  \item[\textbf{Quadrotor:}] aeronave de despegue y aterrizaje vertical que es levantado y propulsado por cuatro rotores.
  \item[\textbf{Rapid control prototyping:}] paradigma de validación de sistemas en donde un prototipo físico de la planta interactúa con un modelo matemático o simulación del controlador de esta.
  \item[\textbf{Red neuronal artificial:}] es un sistema informático que busca emular las redes neuronales biológicas a partir de funciones u operaciones matemáticas.
  \item[\textbf{Saturación:}] en el modelo de color HSV, se refiere a la pureza del matiz, representa la distancia al eje de brillo negro-blanco.
  \item[\textbf{Script:}]  es una secuencia de comandos o instrucciones que conforman un programa informático relativamente simple.
  \item[\textbf{Segmentación:}] dentro del campo del procesamiento de imágenes, se refiere al proceso de dividir una imagen en distintas regiones con atributos similares, logrando hacer una distinción clara entre la información de interés y la información no relevante para el análisis.
  \item[\textbf{Sistema operativo:}] es software encargado de gestionar los recursos de hardware de un sistema informático.
  \item[\textbf{Software in the loop:}] paradigma de validación de sistemas en donde la planta y el sistema de control se representan mediante un modelo matemático e interactuar dentro de una simulación.
  \item[\textbf{Terminal de comandos:}] es una interfaz que le permite al usuario interactuar con un sistema de cómputo de forma explícita a base de un conjunto de instrucciones o comandos bien definidos.
  \item[\textbf{Validación:}] en el ámbito del la gestión y desarrollo de proyectos de software se refiere a la evaluación del producto para determinar si cumple con las expectativas y requerimientos definidos por el cliente.
  \item[\textbf{Valor:}] dentro del modelo de color HSV, se refiere al brillo del matiz y representa un desplazamiento vertical en el eje blanco-negro.
  \item[\textbf{Waypoint:}] es un punto de referencia intermedio que conforma una trayectoria o una ruta para el desplazamiento de algún vehículo.
                        
\end{description}
\clearpage

\textbf{Abreviaturas y acr\'onimos}

\begin{description}
\item[API] Application Programming Interface.
\item[HIL] Hardware in the Loop.
\item[SIL] Software in the Loop.
\item[RCP] Rapid Control Prototyping.
\item[RNA] Red Neuronal Artificial.
\item[ML] Machine Learning.
\item[DL] Deep Learning.     
\item[ROS] Robot Operating System 
\end{description}



\tableofcontents
\listoffigures
\listoftables

\chapter{Introducción}

\section{Antecedente históricos}
En diciembre de 1903, Orville Wright realizó el primer vuelo tripulado en la historia de la humanidad; no tuvo que pasar mucho tiempo para que el concepto de vehículo aéreo no tripulado tuviera un auge dentro de la comunidad científica y militar enfocada a la aviación.

Siendo estrictamente correctos, si se toma en consideración los vehículos capaces de generar sustentación y/o que cuentan con un medio para su control, se puede decir que el primer UAV de la historia, fue diseñado por el inglés Douglas Archibald, al fijar un anemómetro en la cuerda de un cometa, con lo cual fue capaz de medir  la velocidad del viento a una altura de aproximadamente 1200 ft. Más tarde, en 1887, Archibald colocó cámaras en otra cometa, con lo cual desarrolló el primer UAV de reconocimiento, en el mundo.

Hablando específicamente de quadrotores, en 1907, Louis Breguet, un pionero francés de la aviación, junto con su hermano Jacques y su profesor Charles Richet, hicieron una demostración del diseño de un giroplano de 4 rotores. Este prototipo contaba con un motor de 30 caballos de fuerza que alimentaba los 4 rotores, cada uno de los cuales tenía 4 propelas y lograba elevarse hasta un máximo de 0.6 m.

Por otro lado, Etienne Oehmichen, un ingeniero francés, fue el primero en experimentar con diseños de aeronaves de ala rotativa. En 1920, construyó y probó 6 diseños, el segundo de ellos tenía 4 motores y 8 propelas; el cuerpo de esta aeronave estaba hecho de tubos de acero y tenía 4 extremidades, en las cuales se alojaban cada uno de sus rotores con 2 propelas cada uno. En su momento, este diseño destacaba en su estabilidad y controlabilidad, y para la mitad de 1920 ya había realizado más de mil vuelos de prueba. En 1924 estableció un récord mundial al volar una distancia horizontal de 360 m.

Después, en 1922 el Dr. George de Bothezt e Ivan Jerome desarrollaron una aeronave con una estructura en forma de equis y rotores de 6 propelas en sus extremidades. Para 1923 habían realizado hasta 100 vuelos de prueba con una altura máxima de 5 m; sin embargo, este diseño era muy complejo y rígido, dificultando su movimiento lateral y suponiendo una  carga de trabajo, para alimentar la maquinaria, demasiado alta para el piloto.

Además, en 1956 se desarrolló el Convertawings Model A, el cual fue pensado para formar parte de una línea de quadrotores grandes para uso civil y militar. Este prototipo contaba con dos motores, que controlan el giro de dos rotores, cada uno, a partir de lo anterior, el control de la aeronave se lograba al variar el empuje proporcionado por los rotores.


\section{Motivación}



\section{Objetivos}
\subsection{Objetivo general}

Proponer e implementar en simulación un algoritmo de detección de compuertas rectangulares mediante visión artificial para la definición y control de trayectoria de un cuadricóptero autónomo virtual.  


\subsection{Objetivo específicos}

\begin{itemize}
    \item Diseñar un algoritmo de visión artificial capaz de identificar compuertas rectangulares 
    \item Diseñar un algoritmo de gestión de trayectorias de vuelo para un cuadricóptero autónomo 
    \item Diseñar un ambiente de simulación en 3D de un circuito de vuelo basado en una carrera de cuadricópteros autónomos. 
    \item Implementar un ambiente de Software in The Loop utilizando los algoritmos y el ambiente de simulación diseñados para verificar su comportamiento en conjunto 
\end{itemize}

\section{Justificación}
Lejos de ser un atractivo visual y un espectáculo con fines de entretenimiento, las competencias de drones autónomos representan el estado del arte de la robótica aplicada a vehículos con sistemas de navegación autónoma.
Lo anterior se debe a que la robótica siempre se ha enfocado a la automatización de los sistemas; es decir, que los robots sean capaces de realizar tareas o recorridos sin necesidad de intervención humana, para esta última parte, se necesita de algoritmos de percepción y navegación, con los cuales los vehículos puedan ubicarse en el espacio a partir de su sistema de sensores con el que cuentan (tales como tecnología a base de láseres, cámaras estereoscópicas, tecnología ultrasónica, etc.) para que después sea capaz de trazar una trayectoria o seguir una ruta previamente definida.   
Lo anterior ha adquirido una robustez bastante significativa en los últimos años, pues existe una gran cantidad de esfuerzos y colaboraciones dedicadas al desarrollo de los mismos, incluso, se han organizado eventos y competencias con el fin de estimular y potenciar el desarrollo de este tipo de sistemas; tal es el caso de la International Conference on Intelligent Robots and systems (IROS) y AlphaPilot, dos eventos de gran magnitud, creados con el objetivo de tratar, demostrar y fomentar los avances que se tienen en el área.

Por otro lado, la implementación de un sistema robótico autónomo no es una tarea sencilla, y debido a la poca competencia en el mercado también adquiere un costo elevado. 
Para que un robot sea capaz de percibir el ambiente a su alrededor y desplazarse por el mismo, es necesario implementar un sistema de software capaz de coordinar la adquisición de datos proveídos por los sensores y el conjunto de actuadores que permiten que el sistema se desplace. Muchas de las soluciones desarrolladas para afrontar este desafío son privadas y no sé comparte con el público en general, además, algoritmos como el filtro de Kalman o un control PID son ampliamente utilizados en este tipo de sistemas, por lo que existe una posibilidad bastante alta de que todas estas soluciones implementen los mismos algoritmos, lo cual conlleva un desperdicio de tiempo y esfuerzo, sin mencionar que la calidad y eficiencia de cada implementación puede variar bastante.
Debido a lo anterior, soluciones de código abierto como ROS (Robot Operating System; un framework de comunicaciones para el manejo y coordinación de procesos en sistemas robóticos), pueden representar el inicio de la implementación de un estándar en el área, pues al ser de software libre permiten que toda la comunidad utilice, mejore e inspeccione los algoritmos ya implementados.

Además, la realización de pruebas con el sistema físico, para verificar y validar los algoritmos desarrollados, representa un costo muy alto en la mayoría de sistemas con los que se trabaja en el área, por lo que también es necesario disponer de algún tipo de simulador que permita realizar las pruebas sin necesidad de utilizar el prototipo físico con el que se trabajará. Existen diferentes paradigmas de simulación en los que se puede simular la planta mediante software, tales como Hardware in the loop (HIL) y software in the loop (SIL). Ambos paradigmas representan una solución al problema planteado, proveyendo resultados muy cercanos a la realidad y con una arquitectura flexible, que permite realizar una gran cantidad de pruebas o incluso entrenar algoritmos relacionados con inteligencia artificial o redes neuronales, una vez más, sin depender del sistema físico. 

A partir de todo lo anterior, en este trabajo se propone el diseño y la simulación de un algoritmo de visión por computadora para la detección de compuertas rectangulares, similares a aquellas utilizadas en las competencias de drones autónomos, para definir la trayectoria de vuelo de un dron autónomo con el fin de que sea capaz de completar un circuito definido.  El entrenamiento e implementación se realizan dentro de un framework de simulación de SIL, y la gestión y comunicación entre procesos se implementan a partir de una arquitectura diseñada en ROS2, todo lo anterior bajo el paradigma de código abierto con el fin de aprovechar las ventajas previamente mencionadas y aportar los esquemas de configuración y diseño a la comunidad.

\vfill



\section{Planteamiento del problema}


\section{Contribuciones}


\section{Metodología}

En primera instancia, se realiza una revisión bibliográfica intensiva acerca del estado del arte en cuanto a drones guiados por visión artificial, con el objetivo de visualizar las soluciones ya implementadas y conceptualizar la arquitectura necesaria para el sistema, sus componentes, los algoritmos de visión artificial empleados y la configuración necesaria para realizar la integración de todo lo anterior.

Posteriormente, se define el esquema general del proyecto estableciendo el algoritmo de visión artificial a utilizar, el ambiente de simulación, la interfaz de comunicación para la adquisición de datos e imágenes provenientes de la simulación, el modelo de dron a simular y las librerías necesarias para integrar el ambiente de simulación.

Establecido lo anterior, se implementa la arquitectura diseñada para el ambiente de simulación y se realizan vuelos manuales con el modelo de dron definido dentro de un circuito de prueba compuesto por compuertas. A partir de lo anterior, se extraen imágenes de la trayectoria de vuelo del dron por medio de una cámara simulada a bordo del modelo del dron; se utilizan las imágenes recopiladas para el entrenamiento del algoritmo de visión artificial.

Cuando el algoritmo de visión artificial proporciona una identificación adecuada del tipo de compuerta utilizada, se implementa el algoritmo con base en la arquitectura definida. Se realiza la validación del algoritmo en otro circuito de vuelo; a lo largo de la simulación, existe un intercambio de información constante entre la simulación y el algoritmo de visión artificial, la simulación envía imágenes obtenidas durante el vuelo del dron y el algoritmo de visión artificial las analiza, de tal forma que es capaz de identificar el centro de la compuerta más cercana y devuelve comandos de vuelo a la simulación para definir una ruta de vuelo que permita que el dron sea capaz de volar a través de la compuerta identificada y finalizar el circuito de forma autónoma.

Se reportan los resultados obtenidos y las posibles mejoras para el proyecto en su conjunto



\section{Límites y alcances}

\subsection{Alcances}
Se implementa la arquitectura de red neuronal convolucional (RNC) DeepPilot, la cual toma capturas de la única cámara a bordo del drone y predice cuatro comandos de vuelo ($\phi,\theta\psi,h$) como salida. La RNC es entrenada a partir de un dataset proveído por los autores de la arquitectura y que contiene un gran conjunto de imágenes obtenidas a partir de simulación, las cuales están asociadas a ciertos comandos de vuelo.    La arquitectura es evaluada dentro de un entorno de simulación realizado en simulador Gazebo 11, en donde se virtualiza un circuito o pista de obstáculos compuesta por compuertas rectangulares de distintas alturas y color sólido, colocadas en distintas posiciones y orientaciones a lo largo del circuito. Se utiliza ROS2 para coordinar el envío de datos entre la simulación de Gazebo 11 y un nodo propio de ROS2 que contiene el algoritmo y arquitectura de DeepPilot. Además, se documenta de forma detallada la configuración realizada para la creación del ambiente de simulación, especificando la integración entre Gazebo, ROS y Python 3 para la evaluación de la arquitectura de DeepPilot. Por último, el proyecto en su conjunto se distribuye bajo el paradigma de código abierto.

\subsection{Límites}
A diferencia de las contribuciones y proyectos más populares dentro de la comunidad de las carreras de drones autónomos, en donde se utiliza ROS1 y Gazebo en su versión 9, en este proyecto se implementa la última versión estable de ROS2, Foxy, y la versión más actual del simulador Gazebo, al momento de escritura del trabajo, la versión 11. Por lo que es muy posible que algunos plugins tanto de ROS como de Gazebo, no se encuentren disponibles en estas versiones, lo que significa una limitante para la expansión a futuro del proyecto.  Por otro lado, dentro del ambiente de simulación, no se evalúan condiciones de vuelo poco ventajosas como viento en contra, lluvia o cualquier otra condición climática adversa. Además, la complejidad en el arreglo de compuertas para el circuito es baja y se asume que las compuertas se encuentran de forma paralela a la cámara del dron, y  es necesario que siempre exista una compuerta visible después de haber cruzado por otra.


\section{Estructura de la tesis}








\chapter{Estado del Arte}

Las competencias de drones autónomos han adquirido un grado alto de relevancia en la última década, dentro del marco teórico del presente trabajo se describe con profundidad el contexto histórico, así como la motivación y los requerimientos establecidos para dos de las competencias, más significativas, de drones autónomos, el  IROS Autonomous Drone Race y el AlphaPilot AI Drone Innovation Challege. 
En este capítulo se presentan algunas de las soluciones propuestas en estas competencias, al igual que trabajos con enfoques más prácticos o que no se encuentran directamente relacionados con las carreras de drones autónomos.

En cuanto a detección de compuertas

En \cite{cabrera2019gate} se propone un algoritmo de visión artificial basado en una red neuronal profunda con arquitectura de Single Shot Detection (SSD7); se eliminan la ultimas dos capas para reducir el tiempo de predicción y se compara su desempeño con otras arquitecturas: YOLO3, tinyYolo y FRNS.

\chapter{Marco Teórico}

\section{Competencias de Drones Autónomos}

Las carreras de drones se han convertido en un deporte bastante popular en los últimos años. Resulta increíble pensar que, haciendo uso de única y exclusivamente una cámara de vuelo, los pilotos son capaces de abstraer la información necesaria del ambiente para ejecutar maniobras de vuelo con alta precisión y agilidad. 

A partir de lo anterior, la comunidad científica, en especial aquella dedicada al campo de la robótica, se ha visto bastante interesada en sustituir al piloto humano por meras unidades de cómputo y componentes electrónicos; es decir, hoy en día existe la tendencia a automatizar el vuelo de estos vehículos aéreos no tripulados,  de tal manera que, a partir de computadoras de placa única, sensores y algoritmos sofisticados de visión artificial, odometría y gestión y control de trayectoria de vuelo, se pueda obtener el mismo desempeño de vuelo que el otorgado por un piloto humano, e incluso, en algún punto, superarlo de manera significativa.

Sumando a lo ya expresado, se han creado una serie de instituciones y eventos con el fin de financiar, potenciar y motivar el desarrollo tecnológico en este campo emergente, dando lugar a lo que se conoce como \textit{carreras de drones autónomos}. Dentro de los eventos o competencias más significativas se encuentra el \textit{Autonomous Drone Racing (ADR)}\cite{moon2017iros}, llevado a cabo cada año en la Conferencia Internacional de Sistemas y Robots Inteligentes (IROS, por sus siglas en inglés),  el \textit{AlphaPilot Challenge (APC)}\cite{foehn2020alphapilot}, organizado por Lockheed Martin en colaboración con Nvidia y la Liga de Carrera de Drones (DRL); y \textit{Game of Drones (GOD)}\cite{madaan2020airsim}, gestionada por Microsoft para la Conferencia Anual de Sistemas de Procesamiento de Información Neuronal (NeurIPS) de 2019.

Los eventos anteriores representan un punto de encuentro a nivel internacional que ha permitido dirigir los esfuerzo e intelectos alrededor del mundo, a la propuesta de soluciones, ya sea de forma parcial o general, para el dilema ya expresado; y de hecho, es ahí en donde se ha presentado el estado del arte de este enfoque, pues se busca poner a prueba las implementaciones propuestas por los participantes en circuitos y retos con distintas características y composición. En las siguientes subsecciones se describe con más detalle las características, requisitos y relevancia de cada una de las competencias mencionadas.

\subsection{Autonomous Drone Racing}
A grandes rasgos, el ADR es una competencia que busca promover soluciones para vuelos autónomos ágiles en ambientes angostos de interiores. En el desafío se combinan técnicas y enfoques que buscan optimizar distintos parámetros de desempeño, como la generación de trayectoria de vuelo, el tiempo de recorrido de los circuitos, esquemas de control, detección de obstáculos, localización y mapeo, entre otros. 

El ADR debuto como evento en la edición de 2016 del IROS, en Daejeon, Corea. A partir de entonces siguió teniendo presencia en 3 ediciones más del IROS; en 2017 con sede en Vancouver, Canadá, en 2018 en Madrid, España y en 2019 Macao, China. Cabe recalcar que el IROS per se sigue llevando a cabo, sin embargo, la última ADR tuvo lugar en la edición 2019 de este evento, posiblemente por las restricciones derivadas en 2020 por la pandemia provocada por el virus SARS-CoV-2; además, la edición 2021 del IROS, se llevó a cabo de forma virtual.  

En general, en cada edición se propuso una única pista, dentro de una zona techada, con 5 pruebas de vuelo para los equipos participantes; velocidad de vuelo en línea recta a través de una serie de compuertas incompleta, vuelo en curva cerrada, recorrido de un circuito en zigzag, recorrido de un circuito en espiral y a través de compuertas cerradas y vuelo por un corredor con obstáculos dinámicos. 

En la edición de 2016, las compuertas fueron identificadas con un número embebido en un código QR, para facilitar su localización. El tamaño de los drones se limitó a un volumen máximo de  1 m x 1 m x 1 m; a los equipos se les compartieron detalles estructurales sobre el circuito antes de la competencia, por lo que les era posible generar mapas les pudieran auxiliar en la navegación del dron. Además, se les permitió el uso de cualquier tipo de sensor, siempre y cuando este estuviera montado en el chasis del vehículo; se utilizaron distintos tipos de sensores para su participación, incluyendo lidares, láseres, radares y sensores ultrasónicos.

En cada edición del ADR, las compuertas utilizadas para delimitar el circuito han conservado un característico color naranja; en cada evento los circuitos cumplieron con los requerimientos y pruebas mencionadas anterior mente, excepto en la edición 2019 en donde el circuito estuvo compuesto por dos grupos de compuertas LED, alfombras con patrones y luces controladas; además, el tamaño de este circuito fue reducido para producir una pista mucho más angosta, con el objetivo de incrementar la dificultad en el desafió \cite{rojas2021board}.   

Para cada equipo, la competencia comenzaba con el despegue del dron de forma manual, este era posicionado en un punto de inicio y en cuanto se diera la señal, los equipos cedían el control del dron a su sistema de piloto automático; es decir, se tenía que suspenderse toda clase de interacción humana con el sistema de vuelo del dron, y permitir que navegara de forma autónoma hasta completar el circuito. 

\subsection{AlphaPilot Challenge}
Como se mencionó, el APC es otra competencia enfocada en las carreras de drones autónomos, fue presentado como un reto de innovación con un gran premio de \$1 millón de dólares para el equipo ganador; la iniciativa fue creada y lanzada al público por Lockheed Martin en conjunto con La Liga de Carrera de drones, en 2019. El objetivo del desafío fue desarrollar un dron completamente autónomo que pudiera navegar por un circuito de vuelo utilizando visión por computadora; a diferencia de otras competencias, el APC no solo buscaba poner a prueba la capacidad de navegación de los sistemas, sino que, se buscaba explotar por completo los sistemas de vuelo, de tal forma que se buscó evaluar también la velocidad de vuelo y agilidad de las maniobras en una pista  más grande y compleja en comparación con la de ADR. 

Entonces, en el APC se buscó implementar soluciones más complejas que permitieran percibir el ambiente del dron por medio del dron y que los sistemas de control de vuelo fueran capaces de llevar al límite la velocidad de navegación de este; el objetivo era claro, se buscaba ampliar el estado del arte y desarrollar implementaciones que pudieran competir con el desempeño de los mejores pilotos humanos. 

Más de 400 equipos participaron en la etapa de selección de esta primera edición del APC, y solamente los mejores 9 equipos clasificaron para poder participar en la competencia. La segunda fase del reto consistió en 3 carreras de clasificación, de donde se seleccionaron a 6 equipos finalistas. La etapa final de la competencia se disputó con un único circuito, donde los equipos compitieron por llevarse el gran premio de \$1 millón de dólares. Los ganadores de cada etapa y carrera de selección fueron filtrados a partir del tiempo que les tomó completar los circuitos; cada participante contó con 3 intentos para completar los circuitos tan rápido como les fuese posible y sin ningún otro competidor o adversario en la pista.

En cada carrera, los drones empezaban en un podio desde donde tenían que despegar y navegar por una secuencia de compuertas con formar distintas y en un orden predefinido. A diferencia del ADR, a los equipos del APC solo seles informo de la estructura de la pista momentos antes de competir; es decir, no podían hacer uso de un mapa detallado para definir la trayectoria de vuelo que tenía que seguir su dron, sino que, tenía que implementar soluciones que se pudieran adaptar en tiempo real a la posición de las compuertas. Se había estimado una longitud de aproximadamente 300 m para cada pista, sin embargo, debido a dificultades técnicas, la pista de mayor longitud fue la de la carrera final, con una longitud aproximada de 74 m\cite{foehn2020alphapilot}.   

Por último, las características del dron utilizado fueron estandarizadas, por lo que a cada equipo se les otorgó el mismo modelo de dron. Además, a todos los equipos se les facilitó una computadora de vuelo \textit{NVIDIA Jetson Xavier}, para la interfaz con los sensores y actuadores de navegación y también fungió como la unidad de procesamiento para llevar a cabo el vuelo autónomo. El arreglo de sensores a bordo de dron se conformó por un par de cámaras estero con  vista frontal a $30^\circ$, una unidad de medición inercial (IMU), un telémetro láser (LRF); la tabla 1 muestra las especificaciones técnicas de los sensores utilizados. Por último, los drones también estaban equipados con un controlador de vuelo encargado de controlar el empuje y la velocidad angular.

\begin{table}
    \centering
    \begin{tabular}{llll}
        \hline
        Sensor & Modelo & Frecuencia & Detalles\\
        \hline
        Cámara & Leopard Imaging IMX 264 & $60 Hz$ & resolución de 1200 x 720\\
        IMU & Bosch BM1088 & $430 Hz$ & rango: $\pm 24 \text{ g}$, $\pm 34.5 \text{rad/s}$\\
         & & & resolución: $7e^{-4}\text{g, } 1e^{-3}\text{rads/s}$\\
        LRF & Garmin LIDAR-Lite v3 & $120 Hz$ & rango: $1-40$ m\\
        & & & resolución: $0.01 \text{ m}$\\
        \hline
    \end{tabular}
    \label{tab:sensors}
    \caption{Especificaciones de sensores utilizados en el APC \cite{foehn2020alphapilot}}
\end{table}

\subsection{Game of Drones}
En la tercera edición de la Conferencia Anual de Sistemas de Procesamiento de Información Neuronal (NeurIPS), en 2019, el equipo de desarrollo de AirSim\cite{foehn2020alphapilot} en conjunto con la Universidad de Stanford y la Universidad de Zúrich buscaron fomentar el avance de las tecnologías utilizadas den las carreras de drones, gestionando la competencia Game of Drones.

De manera similar a las competencias anteriores, el GOD buscó explotar el potencial de los algoritmos de machine learning y visión por computadora, junto con los avances en cuestión de técnicas de planificación de trayectoria, control y estimación de estado de quadrotores; sin embargo, a diferencia de los otros eventos, el GOD se basó completamente en la utilización de un simulador de vuelo con gráficas foto-realistas para la implementación de las propuestas desarrolladas por los equipos. 

El simulador utilizado para la competencia  fue Microsoft AirSim\cite{airsim2017fsr}, el cual fue desarrollado con el objetivo de hacer más accesible el ámbito de las carreras de drones para ingenieros e investigadores, que tiene un conocimiento bastante amplio sobre algoritmos de machine learning e inteligencia artificial, pero que quizás no esten tan familiarizados con el hardware utilizado por estos sistemas de robótica. AirSim se define como un ambiente de simulación para multi-rotores, que integra un motor de físicas ligero, controlador de vuelo, sensores inerciales, y gracias al uso del motor gráfico Unreal Engine (UE), cuenta con un ambiente con gráficos foto-realistas. Además, también ofrece una API (Application Programming Interface) que permite interactuar y comunicarse con los algoritmos de machine learning, y también provee datos sobre el progreso del recorrido, el desempeño del dron y la habilidad de imponer reglas o normas para la carrera, asociadas a infracciones por colisiones y descalificaciones dentro de la competencia.

La competencia se enfocó en control y planificación de trayectoria, visión artificial y evasión de obstáculos (otro dron oponente). Lo anterior se llevó a cabo en tres niveles con base en el enfoque: 

\textbf{Nivel 1} - Planificación de trayectoria: cada circuito estuvo limitado a dos drones a la vez, en donde uno era el perteneciente al equipo participante y el otro era un dron oponente implementado por el staff de Microsoft. El objetivo fue atravesar todas las compuertas en el menor tiempo posible, evitando colisionar con el dron oponente. La posición de las compuertas y de ambos drones fue proveída a través de la API del simulador. El dron oponente contaba con un algoritmo de trayectoria óptima y volaba con una serie de waypoints generados al azar para cruzar por la sección transversal de cada compuerta.

\textbf{Nivel 2} - Percepción: en esta modalidad, la posición de las compuertas contenía ruido, no había dron oponente y la siguiente compuerta a cruzar no siempre se encontraba a la vista; la posición proveída por la API ayudaría a dirigir al dron en la dirección correcta, sin embargo, el vehículo tenía que valerse de su algoritmo de visión artificial para completar el circuito de manera satisfactoria.

\textbf{Nivel 3} - Percepción y planificación de trayectoria: esta modalidad resulto de la combinación de los dos niveles anteriores. A los participantes se les preveía con datos sobre la posición de las compuertas y había un dron adversario; el objetivo era completar el circuito evitando cualquier colisión con el adversario.

Por último, la competencia consistió de dos etapas, una de clasificación y una ronda para los finalistas. Se registraron 117 participantes, pero solamente 16 calificaron para la competencia.


 
\section{Robot Operating System (ROS)}

De acuerdo con su sitio oficial, ROS (del inglés, Robot Operating System) es un conjunto de herramientas y librerías de software para robótica desarrolladas por Open Robotics bajo el paradigma de software libre u open-source. Este entorno de trabajo destaca por contener algoritmos de última generación y herramientas de desarrollo avanzadas,  que permiten la creación, implementación y reutilización de código para todo tipo de proyectos de robótica.

ROS 1, la primera versión del entorno de trabajo, surgió en 2007 como un ambiente de desarrollo para el PR2 robot, un robot de servicio diseñado para trabajar con personas y creado por la empresa The Willow Garage. Sin embargo, los creadores de ROS buscaban que el entorno de trabajo no se viera limitado a un solo modelo de robot, sino que, pudiera ofrecer herramientas de software para más tipos y modelos de robots, por lo que ROS adquirió varias capas de abstracción mediante la implementación de interfaces para el manejo de mensajes, lo que dio lugar a que el software desarrollado mediante ROS pudiera ser reutilizado en más robots.   

Algunas características que destacan en esta etapa temprana de ROS son:
\begin{itemize}
    \item Gestión de un solo robot
    \item Sin requerimientos de aplicación en tiempo real
    \item Excelente conectividad a la red
    \item Usado principalmente en el ámbito académico y de investigación
\end{itemize}

Hoy en día, ROS es utilizado en una amplia gama de robots, desde robots con ruedas y con forma humanoide, hasta brazos industriales, vehículos aéreos y mucho más. Sin embargo, ha pasado bastante tiempo desde el lanzamiento de la primera versión de ROS, y las necesidades y estándares de la industria han cambiado al igual que el paradigma y la filosofía detrás del desarrollo de ROS.
   
A partir de lo anterior, en 2014 una nueva versión de ROS con un enfoque y estructura distinta es anunciada por Open Robotics.  ROS 2 surge como un completo rediseño para lo que había sido el entorno de trabajo hasta entonces, con esta reestructuración se busca cubrir necesidades y funcionalidades que no habían sido consideradas con  ROS1, pero que habían sido exigidas por la comunidad y la industria. Lo anterior dio lugar al desarrollo de un nuevo conjunto de paquetes 

\section{Visión Artificial}

\section{Gazebo}

\section{Software in The Loop}





\chapter{Resultados}

En este capítulo se presentan los resultados obtenidos; además, se documenta de forma detallada el procedimiento realizado para configurar cada uno del software utilizado, lo anterior debido a que este trabajo también busca funcionar como una guía estructurada que permita la réplica del la implementación desarrollada.

\section{Configuración del Sistema}
La implementación del trabajo se realizó en una computadora portátil modelo \textit{Acer Aspire E5-575}. A continuación se anexan las características físicas más relevantes del hardware utilizado.

\begin{table}
    \centering
    \begin{tabular}{||l|l||}
        \hline
        Parámetro & Descripción\\
        \hline
        Procesador & Intel Core i3-7100U; Dual-core 2.40 GHz\\
        Memoria RAM & 12 GB DDR4\\
        Disco duro & 1 TB Toshiba HDD\\
        Coprocesador de gráficos & Intel HD Graphics 620\\
        \hline
    \end{tabular}
    \caption{Características técnicas de la laptop Aspire E5-575}
    \label{tab:specs}
\end{table}

Con respecto al software, se trabajó con las versiones más recientes y técnicamente compatibles del software que integra el sistema. Para el ambiente de simulación se utilizó \textit{Gazebo 11} en conjunto con \textit{Ardupilot} y el modelo ofrecido para SITL de Arducopter; Para la gestión de procesos se utilizó \textit{ROS2 Foxy}. Además, la computadora en donde se implementó el sistema viene por defecto con el sistema operativo \textit{Windows 10}; sin embargo, para poder integrar el software mencionado es necesario utilizar \textit{Ubuntu 20.04.3 LTS (Focal Fossa)}, el cual fue instalado en un disco duro externo.


\section{Sistema de Visión Artificial}

\section{ROS}

\section{Gazebo}




%\bibliographystyle{apalike}
\bibliography{biblio}

\appendix


\end{document}
