\chapter{Conclusión}

El desarrollo de tecnología enfocada a sistemas de navegación autónoma de cuadricópteros ha tenido un gran auge en los últimos años, pues debido a que estos sistemas combinan un tamaño reducido con una agilidad y autonomía de vuelo bastante buena, son ideales para tareas de reconocimiento, búsqueda y rescate, entre otras, en donde es necesario recorrer zonas angostas a gran velocidad. Sin embargo, el fin último del desarrollo de este campo, es diseñar sistemas autónomos que sean capaces de hacer frente a los mejores pilotos humanos, de tal forma que se le saque el mayor provecho posible a los recursos físicos y componentes del vehículo.

Si bien es cierto que hoy en día existe un gran conjunto de esfuerzos desarrollando soluciones para este tipo de sistemas, el estado del arte del campo de estudio todavía no se encuentra a la altura del desempeño de los pilotos profesionales, o bueno, no sin utilizar una gran cantidad de sensores y cámaras para conocer la odometría del vehículo.  

Por otro lado, retomando lo mencionado sobre las competencias de drones autónomos en el marco teórico, la mayoría de los sistemas desarrollados, si no es que todos, pasan por una etapa de validación mediante simulación, e incluso una de las competencias mencionadas se encuentra basada exclusivamente en ambientes simulados. De ahí que exista la necesidad de tener a disposición una variedad de simuladores enfocados a este tipo de aplicaciones.

Entonces, con base en lo ya mencionado, los objetivos y la metodología definida, se puede concluir que se desarrolló de forma satisfactoria un sistema visión artificial y misión de vuelo de un cuadricóptero autónomo basado en una ambiente virtual, que hace uso de tecnologías de última generación, con requerimientos de hardware básicos, completamente gratuito y de código abierto. Los resultados muestran que los algoritmos desarrollados funcionan de manera adecuada o acorde a los criterios y objetivos establecidos para estos.

Hablando acerca del desempeño del algoritmo de visión artificial, se puede decir que la detección se realizó adecuadamente, pues la sintonización del rango de color permitió filtrar de mejor manera el fondo de los fotogramas captados y aislar el contorno de las compuertas del circuito de vuelo; sin embargo, el algoritmo no es perfecto, de hecho es bastante sencillo, pues utiliza una combinación de métodos de apertura y cerradura para realizar la detección tras la conversión del modelo de color del fotograma. Además, a partir de lo observado durante la segunda sintonización del rango de color y de los resultados obtenidos con ROS, fue posible apreciar que para que el algoritmo de detección de mejores resultado, es necesario que la cámara  y la compuerta se encuentren a una distancia relativamente corta, pues no es capaz filtrar el fondo a grandes distancias. Por otro lado, las pruebas hechas dentro de simulación fueron realizadas tomando en cuenta condiciones ideales en el ambiente, como iluminación, viento, ruido, etc., por lo que, es muy probable que en un ambiente real el rendimiento del algoritmo de detección no sea tan eficiente, y más aún si es que existen cambios en la iluminación del ambiente; Sin embargo, tomando en cuenta la simpleza y el bajo consumo de recursos que conlleva la ejecución del algoritmo, los resultados obtenidos fueron más que adecuado para realizar la prueba del framework de simulación. 

En cuanto al desempeño del algoritmo de seguimiento de trayectoria, los resultados observados fueron bastante buenos, ya que el cuadricópteros fue capaz de completar el circuito de vuelo pasando a través de todas y cada una de las compuertas. En este caso no hay observaciones más detalladas, pues el algoritmo se comportó de acuerdo a lo esperado; sin embargo, en cuanto a la librería utilizada para comandar el vuelo si existen algunas observaciones, pues uno de los principales problemas que surgió durante el desarrollo del trabajo fue la naturaleza asíncrona de ArduPilot y el script con el algoritmo de seguimiento de trayectoria, debido a que, a pesar de leer de forma detallada la documentación, no se encontró alguna bandera o mensaje que indicara el momento en el que el piloto automático estaba listo para recibir comandos de vuelo, y a pesar de que la solución implementada funciona, resulta un poco rudimentaria por los ciclos utilizados para la espera de la inicialización de sensores en el piloto automático.

Por último, en cuanto a los problemas enfrentados durante el desarrollo del trabajo, el principal obstáculo fue la curva de aprendizaje de las tecnologías utilizadas, principalmente ROS 2, pues la documentación asociada al framework todavía es algo escasa, y como se mencionó, hoy en día ROS 1 sigue teniendo mucha presencia en las aplicaciones desarrolladas con el framework, de tal forma que no es posible aplicar todo lo documentado por la comunidad para aplicaciones de ROS 2. Además, la documentación de ArduPilot se encuentra bastante desactualizada y también es limitada, por lo que gran parte del trabajo desarrollado para el proyecto ha conllevado una investigación intensiva en foros y otras fuentes poco concurridas, además de experimentación. Entonces, se espera que la documentación proveída en el trabajo sea de gran utilidad para la comunidad, debido a que representa un compilado de muchos fragmentos de información esparcidos en distintas fuentes.

\section{Desarrollos futuros}

Está claro que los resultados obtenidos en este proyecto abren las puertas a una serie de mejoras que permitan explotar de mejor manera lo aquí presentado, y como se mencionó al inicio del documento, este proyecto sienta las bases de una ambiente de simulación que puede ser explotado todavía más allá del alcance logrado en este trabajo. 

Entre las posibles propuestas para proyectos derivados del presente trabajo se tiene:

\begin{itemize}
    \item El diseño de un circuito de vuelo con una trayectoria más compleja.
    \item La implementación de un algoritmo de visión artificial basado en redes neuronales convolucionales.
    \item La integración de los algoritmos de segmentación y seguimiento de trayectoria a partir del uso de servicios en ROS, de tal forma que la aplicación de visión por computadora participe de forma activa para la definición la trayectoria de vuelo del cuadricóptero.
    \item El uso de la librería de MAVROS en lugar de PymavLink para el envío de comandos de vuelo y recepción de información sobre el vehículo, pues es de más bajo nivel y se integra de forma nativa al concepto de nodos en ROS.
    \item La implementación de la arquitectura en una computadora de vuelo real. Debido a que el algoritmo de visión artificial propuesto utiliza funciones básicas de morfología matemática, no requiere el uso de una tarjeta gráfica para ser ejecutado, por lo que el sistema propuesto puede ser implementado en una computadora de placa única, como una Raspberry Pi.
\end{itemize}