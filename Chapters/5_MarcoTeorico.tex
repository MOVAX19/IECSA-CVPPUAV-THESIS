\chapter{Marco Teórico}


\section{Robot Operating System (ROS)}

De acuerdo con su sitio oficial, ROS (del inglés, Robot Operating System) es un conjunto de herramientas y librerías de software para robótica desarrolladas por Open Robotics bajo el paradigma de software libre u open-source. Este entorno de trabajo destaca por contener algoritmos de última generación y herramientas de desarrollo avanzadas,  que permiten la creación, implementación y reutilización de código para todo tipo de proyectos de robótica.

ROS 1, la primera versión del entorno de trabajo, surgió en 2007 como un ambiente de desarrollo para el PR2 robot, un robot de servicio diseñado para trabajar con personas y creado por la empresa The Willow Garage. Sin embargo, los creadores de ROS buscaban que el entorno de trabajo no se viera limitado a un solo modelo de robot, si no que, pudiera ofrecer herramientas de software para más tipos y modelos de robots, por lo que ROS adquirió varias capas de abstracción mediante la implementación de interfaces para el manejo de mensajes, lo que dio lugar a que el software desarrollado mediante ROS pudiera ser reutilizado en más robots.   

Algunas características que destacan en esta etapa temprana de ROS son:
Gestión de un solo robot
Sin requerimientos de aplicación en tiempo real
Excelente conectividad a la red
Usado principalmente en el ámbito académico y de investigación

Hoy en día, ROS es utilizado en una amplia gama de robots, desde robots con ruedas y con forma humanoide, hasta brazos industriales, vehículos aéreos y mucho más. Sin embargo, ha pasado bastante tiempo desde el lanzamiento de la primera versión de ROS, y las necesidades y estándares de la industria han cambiado al igual que el paradigma y la filosofía detrás del desarrollo de ROS.
   
A partir de lo anterior, en 2014 una nueva versión de ROS con un enfoque y estructura distinta es anunciada por Open Robotics.  ROS 2 surge como un completo rediseño para lo que había sido el entorno de trabajo hasta entonces, con esta reestructuración se busca cubrir necesidades y funcionalidades que no habían sido consideradas con  ROS 1 pero que habían sido exigidas por la comunidad y la industria. Lo anterior dio lugar al desarrollo de un nuevo conjunto de paquetes  
