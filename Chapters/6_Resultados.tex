\chapter{Resultados}

En este capítulo se presentan los resultados obtenidos; además, se documenta de forma detallada el procedimiento realizado para configurar cada uno del software utilizado, lo anterior debido a que este trabajo también busca funcionar como una guía estructurada que permita la réplica del la implementación desarrollada.

\section{Configuración del Sistema}
La implementación del trabajo se realizó en una computadora portátil modelo \textit{Acer Aspire E5-575}. A continuación se anexan las características físicas más relevantes del hardware utilizado.

\begin{table}
    \centering
    \begin{tabular}{||l|l||}
        \hline
        Parámetro & Descripción\\
        \hline
        Procesador & Intel Core i3-7100U; Dual-core 2.40 GHz\\
        Memoria RAM & 12 GB DDR4\\
        Disco duro & 1 TB Toshiba HDD\\
        Coprocesador de gráficos & Intel HD Graphics 620\\
        \hline
    \end{tabular}
    \caption{Características técnicas de la laptop Aspire E5-575}
    \label{tab:specs}
\end{table}

Con respecto al software, se trabajó con las versiones más recientes y técnicamente compatibles del software que integra el sistema. Para el ambiente de simulación se utilizó \textit{Gazebo 11} en conjunto con \textit{Ardupilot} y el modelo ofrecido para SITL de Arducopter; Para la gestión de procesos se utilizó \textit{ROS2 Foxy}. Además, la computadora en donde se implementó el sistema viene por defecto con el sistema operativo \textit{Windows 10}; sin embargo, para poder integrar el software mencionado es necesario utilizar \textit{Ubuntu 20.04.3 LTS (Focal Fossa)}, el cual fue instalado en un disco duro externo.


\section{Sistema de Visión Artificial}

\section{ROS}

\section{Gazebo}

