\chapter{Resultados}

En este capítulo se presentan los resultados obtenidos; además, se documenta de forma detallada el procedimiento realizado para configurar cada uno del software utilizado, lo anterior debido a que este trabajo también busca funcionar como una guía estructurada que permita la réplica del la implementación desarrollada.

\section{Configuración del Sistema}
La implementación del trabajo se realizó en una computadora portátil modelo \textit{Acer Aspire E5-575}. A continuación se anexan las características físicas más relevantes del hardware utilizado.

\begin{table}
    \centering
    \begin{tabular}{||l|l||}
        \hline
        Parámetro & Descripción\\
        \hline
        Procesador & Intel Core i3-7100U; Dual-core 2.40 GHz\\
        Memoria RAM & 12 GB DDR4\\
        Disco duro & 1 TB Toshiba HDD\\
        Coprocesador de gráficos & Intel HD Graphics 620\\
        \hline
    \end{tabular}
    \caption{Características técnicas de la laptop Aspire E5-575}
    \label{tab:specs}
\end{table}

Con respecto al software, se trabajó con las versiones más recientes y técnicamente compatibles del software que integra el sistema. Para el ambiente de simulación se utilizó \textit{Gazebo 11} en conjunto con \textit{Ardupilot} y el modelo ofrecido para SITL de Arducopter; Para la gestión de procesos se utilizó \textit{ROS2 Foxy}. Además, la computadora en donde se implementó el sistema viene por defecto con el sistema operativo \textit{Windows 10}; sin embargo, para poder integrar el software mencionado es necesario utilizar \textit{Ubuntu 20.04.3 LTS (Focal Fossa)}, el cual fue instalado en un disco duro externo.

\subsection{Instalación de ROS 2}
La siguiente serie de comandos fue extraída de la documentación oficial de ROS2 Foxy\cite{ros2} y se asume que la instalación se lleva a cabo en un sistema con Ubuntu 20.04 o sus derivados (\textit{Xubuntu, Kubuntu}, etc.). El proceso puede ser distinto para cualquier otra distribución de Linux o Sistema operativo no listado en la documentación oficial, o incluso puede que no sea compatible.

\begin{enumerate}
    \item Revisar que el sistema donde se instalará ROS2 admite la codificación de caracteres \textit{UTF-8}, mediante el siguiente comando.

    \begin{lstlisting}[language = bash]
        $ locale
    \end{lstlisting}

    Sí la codificación se encuentra en la lista, se puede saltar al paso x, si no, seguir se debe seguir con el resto de pasos.

    \item Instalar la codificación de caracteres

    \begin{lstlisting}[language = bash]
        $ sudo apt update && sudo apt install locales
        $ sudo locale-gen en_US en_US.UTF-8
        $ sudo update-locale LC_ALL=en_US.UTF-8 LANG=en_US.UTF-8
        $ export LANG=en_US.UTF-8
        $ locale #verificacion de instalacion
    \end{lstlisting}

    \item Añadir el repositorio de ROS 2 al sistema. 
    
    \begin{lstlisting}[language = bash]
        $ sudo apt update && sudo apt install curl gnupg2 lsb-release
        $ sudo curl -sSL https://raw.githubusercontent.com/ros/rosdistro/master/ros.key  -o /usr/share/keyrings/ros-archive-keyring.gpg
        $ echo "deb [arch=$(dpkg --print-architecture) signed-by=/usr/share/keyrings/ros-archive-keyring.gpg] http://packages.ros.org/ros2/ubuntu $(lsb_release -cs) main" | sudo tee /etc/apt/sources.list.d/ros2.list > /dev/null
    \end{lstlisting}

    \item Instalar las herramientas de desarrollo para ROS 2

    \begin{lstlisting}[language = bash]
        $ sudo apt update && sudo apt install -y \
          build-essential \
          cmake \
          git \
          libbullet-dev \
          python3-colcon-common-extensions \
          python3-flake8 \
          python3-pip \
          python3-pytest-cov \
          python3-rosdep \
          python3-setuptools \
          python3-vcstool \
          wget
        # Paquetes de Python 3 para pruebas
        $ python3 -m pip install -U \
          argcomplete \
          flake8-blind-except \
          flake8-builtins \
          flake8-class-newline \
          flake8-comprehensions \
          flake8-deprecated \
          flake8-docstrings \
          flake8-import-order \
          flake8-quotes \
          pytest-repeat \
          pytest-rerunfailures \
          pytest
        # Dependencias Fast-RTPS
        $ sudo apt install --no-install-recommends -y \
          libasio-dev \
          libtinyxml2-dev
        # Dependencias Cyclone DDS
        $ sudo apt install --no-install-recommends -y \
          libcunit1-dev
    \end{lstlisting}

    \item Clonar el código fuente de ROS 2

    \begin{lstlisting}[language = bash]
        $ mkdir -p ~/ros2_foxy/src #crea el ambiente de trabajo
        $ cd ~/ros2_foxy
        $ wget https://raw.githubusercontent.com/ros2/ros2/foxy/ros2.repos
        $ vcs import src < ros2.repos
    \end{lstlisting}

    \item Instalar dependencias

    \begin{lstlisting}[language = bash]
        $ sudo rosdep init
        $ rosdep update
        $ rosdep install --from-paths src --ignore-src -y --skip-keys "fastcdr rti-connext-dds-5.3.1 urdfdom_headers"
    \end{lstlisting}

    \item Compilar código fuente
    
    \begin{lstlisting}[language = bash]
        $ cd ~/ros2_foxy/
        $ colcon build --symlink-install
    \end{lstlisting}

    \item Habilitar el API de ROS 2 en bash
    
    \begin{lstlisting}[language = bash]
        $ source /opt/ros/foxy/setup.bash
    \end{lstlisting}

    \item Modificar el perfil de bash para que inicie ROS 2 con cada nueva terminal
    
    \begin{lstlisting}[language = bash]
        $ echo "source /opt/ros/foxy/setup.bash" >> ~/.bashrc 
    \end{lstlisting}

\end{enumerate}

Hecho lo anterior, el sistema debe de contar con una instalación completa de ROS 2. Para comprobar que la instalación se llevó a cabo de manera correcta, se pueden ejecutar los nodos demo que vienen incluidos en la instalación de escritorio.


\section{Sistema de Visión Artificial}

\section{ROS}

\section{Gazebo}

