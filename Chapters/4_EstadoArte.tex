\chapter{Estado del Arte}

Las competencias de drones autónomos han adquirido un grado alto de relevancia en la última década, dentro del marco teórico del presente trabajo se describe con profundidad el contexto histórico, así como la motivación y los requerimientos establecidos para dos de las competencias, más significativas, de drones autónomos, el  IROS Autonomous Drone Race y el AlphaPilot AI Drone Innovation Challege. 
En este capítulo se presentan algunas de las soluciones propuestas en estas competencias, al igual que trabajos con enfoques más prácticos o que no se encuentran directamente relacionados con las carreras de drones autónomos.

Dentro de las competencias anteriormente mencionadas, existen dos problemas esenciales a los que se enfrentan los equipos que participan en estos retos, la detección de objetos y la gestión de trayectoria de vuelo a partir de la detección realizada. 
Los circuitos que tiene que completar los drones están compuestos por compuertas de distintas formas y tamaños, y en algunos casos, se adicionan obstáculos dinámicos, los vehículos desarrollados por los participantes tienen que ser capaces de detectar estos objetos haciendo uso exclusivo de los sensores con los que están equipados (cámaras, sensores ultrasónicos, tecnología láser, etc.).

En uno de sus artículos, cabrera 

En \cite{cabrera2019gate} se propone un algoritmo de visión artificial basado en una red neuronal profunda con arquitectura de Single Shot Detection (SSD7); se eliminan la ultimas dos capas para reducir el tiempo de predicción y se compara su desempeño con otras arquitecturas: YOLO3, tinyYolo y FRNS.

En \cite{moon2019challenges} se presenta la motivación detrás de la creación de las competencias de drones autónomos, así como los algoritmos implementados en la edición 2018 del IROS 

