\chapter{Estado del Arte}

Las competencias de drones autónomos han adquirido un grado alto de relevancia en la última década, dentro del marco teórico del presente trabajo se describe con profundidad el contexto histórico, así como la motivación y los requerimientos establecidos para dos de las competencias, más significativas, de drones autónomos, el  IROS Autonomous Drone Race y el AlphaPilot AI Drone Innovation Challege. 
En este capítulo se presentan algunas de las soluciones propuestas en estas competencias, al igual que trabajos con enfoques más prácticos o que no se encuentran directamente relacionados con las carreras de drones autónomos.

Dentro de las competencias anteriormente mencionadas, existen dos problemas esenciales a los que se enfrentan los equipos que participan en estos retos, la detección de objetos y la gestión de trayectoria de vuelo a partir de la detección realizada. 
Los circuitos que tiene que completar los drones están compuestos por compuertas de distintas formas y tamaños, y en algunos casos, se adicionan obstáculos dinámicos, los vehículos desarrollados por los participantes tienen que ser capaces de detectar estos objetos haciendo uso exclusivo de los sensores con los que están equipados (cámaras, sensores ultrasónicos, tecnología láser, etc.).

Con base en lo anterior, Cabrera et al.(2019). \cite{cabrera2019gate} desarrollaron un algoritmo para la detección de compuertas en tiempo real basado en aprendizaje profundo. Su implementación se basó en una arquitectura de red neuronal convolucional con una arquitectura base de Single Shot Detector de 7 capas (SSD7\cite{SSD7}). La arquitectura base tiene un diseño optimizado para la detección de objetos, permitiendo un tiempo de entrenamiento reducido y una velocidad de detección alta; esta se modificó de tal forma que se eliminaron las últimas dos capas convoluciones, haciendo posible una detección mucho más rápida que la propuesta base y disminuyendo la complejidad de la red. El entrenamiento de la red se realizó con un total de 3418 imágenes obtenidas a partir de un entorno simulado y entornos reales. 
Además, para observar el desempeño de su implementación compararon su arquitectura con otras propuestas, SSD7, SSD300 y SmallerVGG, en simulaciones y ambientes de exteriores e interiores. Los resultados muestran que su propuesta logra un tiempo de detección promedio más bajo y porcentaje de confianza más alto que las otras arquitecturas. 

Por otro lado, Mellinger y Kumar (2011)\cite{mellinger2011minimum} presentaron un diseño de control y generación de trayectoria de vuelo en ambientes de interiores para un quadrotor. Su implementación es capaz de generar una trayectoria óptima y ángulos para la guiñada del vehículo, en tiempo real, a partir de matrices de rotación para el marco de referencia del vehículo y una secuencia de posiciones en tres dimensiones. La propuesta fue diseñada con el objetivo de que el quadrotor sea capaz de navegar de forma segura a través de corredores angostos, manteniéndose en los límites de velocidad y aceleración. Además, implementaron un control no lineal que asegura el seguimiento de las trayectorias generadas; las propuestas se pusieron a prueba con un prototipo físico  que se hizo volar a través de un circuito construido por aros, los cuales indicaban la trayectoria que el quadrotor debía de seguir.

A demás, Mueller et al.(2013)\cite{mueller2013computationally} diseñaron un algoritmo de bajo consumo computacional para la generación de trayectorias de intersección vuelo de un quadrotor. La implementación tuvo como propósito que el quadrotor fuera capaz de interceptar una pelota en vuelo, con una raqueta montada en su chasis. El algoritmo de generación de trayectoria se usó en un sistema de control predictivo, en donde miles de trayectorias eran generadas y evaluadas por el controlador, y después, la trayectoria más óptima era seleccionada por el algoritmo.  Se destaca el bajo coste computacional pues se utilizó el hardware de una laptop estándar para evaluar cerca de un millón de trayectorias por segundo.

Las propuestas anteriores representan ejemplos de soluciones individuales para cada uno de los problemas mencionados. Sin embargo, existen implementaciones que solucionan ambos problemas en un solo trabajo, y corresponden a aquellas que fueron desarrolladas como propuestas para participar en las competencias.  

