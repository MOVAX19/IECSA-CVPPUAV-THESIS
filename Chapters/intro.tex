\chapter{Introducción}

\section{Motivación}



\section{Objetivos}
\subsection{Objetivo general}

Proponer e implementar en simulación un algoritmo de detección de compuertas rectangulares mediante visión artificial para la definición y control de trayectoria de un cuadricóptero autónomo virtual.   

\subsection{Objetivo específicos}

\begin{itemize}
    \item Diseñar un algoritmo de visión artificial capaz de identificar compuertas rectangulares 
    \item Diseñar un algoritmo de gestión de trayectorias de vuelo para un cuadricóptero autónomo 
    \item Diseñar un ambiente de simulación en 3D de un circuito de vuelo basado en una carrera de cuadricópteros autónomos. 
    \item Implementar un ambiente de Software in The Loop utilizando los algoritmos y el ambiente de simulación diseñados para verificar su comportamiento en conjunto 
\end{itemize}

\section{Justificación}
Lejos de ser un atractivo visual y un espectáculo con fines de entretenimiento, las competencias de drones autónomos representan el estado del arte de la robótica aplicada a vehículos con sistemas de navegación autónoma.
Lo anterior se debe a que la robótica siempre se ha enfocado a la automatización de los sistemas; es decir, que los robots sean capaces de realizar tareas o recorridos sin necesidad de intervención humana, para esta última parte, se necesita de algoritmos de percepción y navegación, con los cuales los vehículos puedan ubicarse en el espacio a partir de su sistema de sensores con el que cuentan (tales como tecnología a base de láseres, cámaras estereoscópicas, tecnología ultrasónica, etc.) para que después sea capaz de trazar una trayectoria o seguir una ruta previamente definida.   
Lo anterior ha adquirido una robustez bastante significativa en los últimos años, pues existe una gran cantidad de esfuerzos y colaboraciones dedicadas al desarrollo de los mismos, incluso, se han organizado eventos y competencias con el fin de estimular y potenciar el desarrollo de este tipo de sistemas; tal es el caso de la International Conference on Intelligent Robots and systems (IROS) y AlphaPilot, dos eventos de gran magnitud, creados con el objetivo de tratar, demostrar y fomentar los avances que se tienen en el área.

Por otro lado, la implementación de un sistema robótico autónomo no es una tarea sencilla, y debido a la poca competencia en el mercado también adquiere un costo elevado. 
Para que un robot sea capaz de percibir el ambiente a su alrededor y desplazarse por el mismo, es necesario implementar un sistema de software capaz de coordinar la adquisición de datos proveídos por los sensores y el conjunto de actuadores que permiten que el sistema se desplace. Muchas de las soluciones desarrolladas para afrontar este desafío son privadas y no sé comparte con el público en general, además, algoritmos como el filtro de Kalman o un control PID son ampliamente utilizados en este tipo de sistemas, por lo que existe una posibilidad bastante alta de que todas estas soluciones implementen los mismos algoritmos, lo cual conlleva un desperdicio de tiempo y esfuerzo, sin mencionar que la calidad y eficiencia de cada implementación puede variar bastante.
Debido a lo anterior, soluciones de código abierto como ROS (Robot Operating System; un framework de comunicaciones para el manejo y coordinación de procesos en sistemas robóticos), pueden representar el inicio de la implementación de un estándar en el área, pues al ser de software libre permiten que toda la comunidad utilice, mejore e inspeccione los algoritmos ya implementados.

Además, la realización de pruebas con el sistema físico, para verificar y validar los algoritmos desarrollados, representa un costo muy alto en la mayoría de sistemas con los que se trabaja en el área, por lo que también es necesario disponer de algún tipo de simulador que permita realizar las pruebas sin necesidad de utilizar el prototipo físico con el que se trabajará. Existen diferentes paradigmas de simulación en los que se puede simular la planta mediante software, tales como Hardware in the loop (HIL) y software in the loop (SIL). Ambos paradigmas representan una solución al problema planteado, proveyendo resultados muy cercanos a la realidad y con una arquitectura flexible, que permite realizar una gran cantidad de pruebas o incluso entrenar algoritmos relacionados con inteligencia artificial o redes neuronales, una vez más, sin depender del sistema físico. 

A partir de todo lo anterior, en este trabajo se propone el diseño y la simulación de un algoritmo de visión por computadora para la detección de compuertas rectangulares, similares a aquellas utilizadas en las competencias de drones autónomos, para definir la trayectoria de vuelo de un dron autónomo con el fin de que sea capaz de completar un circuito definido.  El entrenamiento e implementación se realizan dentro de un framework de simulación de SIL, y la gestión y comunicación entre procesos se implementan a partir de una arquitectura diseñada en ROS2, todo lo anterior bajo el paradigma de código abierto con el fin de aprovechar las ventajas previamente mencionadas y aportar los esquemas de configuración y diseño a la comunidad.

\vfill

\section{Justificación}


\section{Planteamiento del problema}


 \section{Contribuciones}


\section{Límites y alcances}

\subsection{Alcances}


\subsection{Límites}



\section{Estructura de la tesis}






