\chapter*{Glosario}

\textbf{  \normalsize Terminología}
\begin{description}
  \item[\textbf{Algoritmo:}] conjunto finito y ordenado de instrucciones que representan la solución a un problema.
  \item[\textbf{Aprendizaje profundo:}] del inglés '\textit{Deep Learning}', es una subárea de la inteligencia artificial, en donde el modelo de aprendizaje se basa en una gran conjunto de capas (de entrada, salidas y ocultas) compuestas por redes neuronales artificiales. Cada capa se especializa una tarea de predicción específica, de tal forma que una máquina es capaz de aprender por sí misma, sin necesidad de intervención humana.
  \item[\textbf{Arquitectura:}] dentro del campo de estudio de la inteligencia artificial, se refiere a las conexiones o el patrón de diseño de una red neuronal artificial.
  \item[\textbf{Código abierto:}] también conocido como \textit{software libre}, es un modelo de desarrollo de software que se fundamenta en la colaboración abierta, en donde cualquier usuario tiene la liberta de ejecutar, copiar, distribuir, modificar y contribuir a la mejora del software. 
  \item[\textbf{Comando de vuelo:}] dentro del contexto del firmware para pilotos automáticos, se refiere a instrucciones de alto nivel para que el piloto automático lleve a la aeronave a un estado deseado, dígase actitud, rumbo, etc.
  \item[\textbf{Convolución:}] operador matemático que representa la integral del producto de dos funciones, en donde una de las señales se encuentra trasladada e invertida. 
  \item[\textbf{Entrenamiento:}] en el campo de estudio de la inteligencia artificial, se refiere al conjunto de métodos a partir de los cuales una máquina es capaz de aprender.
  \item[\textbf{Espacio de color:}] también conocido como \textit{modelo de color}, se refiere a los distintos sistemas mediante los cuales se pueden representar los colores.
  \item[\textbf{Firmware:}] es el software base que viene incluido en los dispositivos electrónicos o hardware, y se encarga de asegurar un funcionamiento básico correcto. También es conocido como \textit{soporte lógico inalterable}.
  \item[\textbf{Fotograma:}] cada una de las imágenes fijas, que en su conjunto forman una imagen en movimiento o video.
  \item[\textbf{Framework:}] conocido en español como \textit{entorno de trabajo}, es una estructura que integra tecnologías, estándares y módulos de software que sirve como base para el desarrollo software.
  \item[\textbf{Hardware:}] corresponde a los recursos físicos que componen o integran un equipo de cómputo o dispositivo lógico.
  \item[\textbf{Hardware in the loop:}] paradigma de simulación para la validación de sistemas en donde el proceso que se desea controlar se simula a partir de un modelo matemático, mientras que el sistema de control es físico e interactúa con la simulación de la planta por medio de una estación de trabajo.
  \item[\textbf{Histograma de color:}] es la cuantificación de la distribución de color en una imagen, generalmente se representa con una gráfica en donde se observa la frecuencia de pixeles del mismo color.
  \item[\textbf{Interfaz de programación de aplicación:}] del inglés \textit{Application} \textit{Programming} \textit{Interface}; se trata de un conjunto de definiciones y protocolos que permiten la comunicación o integración entre diferentes aplicaciones de software.
  \item[\textbf{Librería:}] también conocidas como \textit{bibliotecas}, es un conjunto de módulos o métodos funcionales de software, que fueron codificados para ofrecer una funcionalidad especifica y bien definida.
  \item[\textbf{Machine learning:}] conocido en español como \textit{aprendizaje automático}, es una subárea del campo de la inteligencia artificial en donde se implementa modelo matemático para que un sistema sea capaz de aprender a partir del procesamiento de datos sin la necesidad de especificar una programación explicita.
  \item[\textbf{Máquina de estados:}] es un modelo que describe el comportamiento de un sistema a partir de una serie de estados finitos, en donde la transición entre cada uno depende de la entrada actual del proceso y la o las entradas anteriores.
  \item[\textbf{Matiz:}] en el modelo de color HSV, corresponde a un ángulo dentro del rango de 0 a 360 grados, en donde cada grado está asociado a una tonalidad de color en específico.
  \item[\textbf{Multiplataforma:}] dicho de una aplicación de software que se encuentra disponible para su ejecución en distintos sistemas operativos o sistemas.
  \item[\textbf{Odometría:}] es el área que se encarga del estudio de la estimación de posición de cualquier tipo de vehículo durante su navegación.
  \item[\textbf{Quadrotor:}] aeronave de despegue y aterrizaje vertical que es levantado y propulsado por cuatro rotores.
  \item[\textbf{Rapid control prototyping:}] paradigma de validación de sistemas en donde un prototipo físico de la planta interactúa con un modelo matemático o simulación del controlador de esta.
  \item[\textbf{Red neuronal artificial:}] es un sistema informático que busca emular las redes neuronales biológicas a partir de funciones u operaciones matemáticas.
  \item[\textbf{Saturación:}] en el modelo de color HSV, se refiere a la pureza del matiz, representa la distancia al eje de brillo negro-blanco.
  \item[\textbf{Script:}]  es una secuencia de comandos o instrucciones que conforman un programa informático relativamente simple.
  \item[\textbf{Segmentación:}] dentro del campo del procesamiento de imágenes, se refiere al proceso de dividir una imagen en distintas regiones con atributos similares, logrando hacer una distinción clara entre la información de interés y la información no relevante para el análisis.
  \item[\textbf{Sistema operativo:}] es software encargado de gestionar los recursos de hardware de un sistema informático.
  \item[\textbf{Software in the loop:}] paradigma de validación de sistemas en donde el proceso y el sistema de control se representan mediante un modelo matemático e interactuan dentro de una simulación.
  \item[\textbf{Terminal de comandos:}] es una interfaz que le permite al usuario interactuar con un sistema de cómputo de forma explícita a base de un conjunto de instrucciones o comandos bien definidos.
  \item[\textbf{Validación:}] en el ámbito del la gestión y desarrollo de proyectos de software se refiere a la evaluación del producto para determinar si cumple con las expectativas y requerimientos definidos por el cliente.
  \item[\textbf{Valor:}] dentro del modelo de color HSV, se refiere al brillo del matiz y representa un desplazamiento vertical en el eje blanco-negro.
  \item[\textbf{Waypoint:}] es un punto de referencia intermedio que conforma una trayectoria o una ruta para el desplazamiento de algún vehículo.
                        
\end{description}
\clearpage

\textbf{Abreviaturas y acr\'onimos}

\begin{description}
\item[API] Application Programming Interface.
\item[HIL] Hardware in the Loop.
\item[SIL] Software in the Loop.
\item[RCP] Rapid Control Prototyping.
\item[RNA] Red Neuronal Artificial.
\item[ML] Machine Learning.
\item[DL] Deep Learning.     
\item[ROS] Robot Operating System 
\end{description}


